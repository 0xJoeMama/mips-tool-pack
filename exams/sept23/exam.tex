\documentclass{article}
\usepackage[top=2.5cm, left=3cm, right=3cm, bottom=4.0cm]{geometry}
\usepackage{fullpage}
\usepackage{latexsym}
\usepackage{amsmath}
\usepackage{amssymb}
\usepackage{amsthm}
\usepackage{epsfig}
\usepackage[utf8]{inputenc}
\usepackage[greek,english]{babel}
\usepackage{alphabeta}
\usepackage{kerkis}
\usepackage{enumerate}

\title{Εξέταση εργαστηρίου Αρχιτεκτονική Υπολογιστών Ι}
\author{}
\date{Σεπτέμβριος 2023}
\begin{document}
\maketitle
%%%%%%%%%%%%%%%%%
%   Problem 1α  %
%%%%%%%%%%%%%%%%%
\setcounter{equation}{0}
\section*{Exam 1}
Γράψτε πρόγραμμα assembly MIPS που συνεχώς:
\begin{enumerate}[i.]
    \itemsep0em
    \item Θα εκτυπώνει: "Enter lastname and firstname as a string of $30$ chars: ",
    \item Θα αλλάζει γραμμή,
    \item Θα δέχεται ως είσοδο το ονοματεπώνυμο του φοιτητή ως μία συμβολοσειρά $30$ το πολύ χαρακτήρων και θα την εκτυπώνει,
    \item Θα αλλάζει γραμμή,
    \item Θα εκτυπώνει: "Enter real number: ",
    \item Θα διαβάζει έναν πραγματικό αριθμό απλής ακρίβειας, έστω $r$, από την κονσόλα και θα τον εκτυπώνει,
    \item Θα αλλάζει γραμμή,
    \item Αν είναι $r=0.0$ θα εκτυπώνει "End of program."
    \item Αλλιώς, θα εκτυπώνει: "Reversing the bits of real $r$:"
    \item Θα αντιστρέφει την σειρά των bits του πραγματικού και θα εκτυπώνει τον πραγματικό που προκύπτει. Δηλαδή το bit$0$ θα γίνει bit$31$, το bit$1$ θα γίνει bit$30$ κοκ.
    \item Θα συνεχίζει την ίδια διαδικασία μέχρι να αναγνωστεί $r=0.0$.
\end{enumerate}
\newpage
\section*{Exam 2}
Γράψτε πρόγραμμα assembly MIPS που συνεχώς:
\begin{enumerate}[i.]
    \itemsep0em
    \item Θα εκτυπώνει: "Enter lastname and firstname as a string of $30$ chars: ",
    \item Θα αλλάζει γραμμή,
    \item Θα δέχεται ως είσοδο το ονοματεπώνυμο του φοιτητή ως μία συμβολοσειρά $30$ το πολύ χαρακτήρων και θα την εκτυπώνει,
    \item Θα αλλάζει γραμμή,
    \item Θα εκτυπώνει: "Enter real number: ",
    \item Θα διαβάζει έναν πραγματικό αριθμό απλής ακρίβειας, έστω $r$, από την κονσόλα και θα τον εκτυπώνει,
    \item Θα αλλάζει γραμμή,
    \item Αν είναι $r=0.0$ θα εκτυπώνει "End of program."
    \item Διαφορετικά θα ελέγχει αν ο $r$ είναι ακέραιος και δύναμη του $2$ ή $\pm$inf.
    Αν είναι τότε θα εκτυπώνει "Integer and power of 2 or $\pm$inf!", διαφορετικά θα εκτυπώνει
    "Not integer and power of 2, not $\pm$inf!".
    \item Θα συνεχίζει την ίδια διαδικασία μέχρι να αναγνωστεί $r=0.0$.
\end{enumerate}
Ο ζητούμενος έλεγχος να γίνεται από το περιεχόμενο του Exponent και της Mantissa.

\section*{Exam 3}
Ο πιλότος ενός σύγχρονου αεροπλάνου αερογραμμών περνάει στο FMS (Flight Management System)
τις συντεταγμένες ενός σημείου που δεν υπάρχει ήδη ως έναν 11/ψήφιο της μορφής 3827N02308E ή
3827S02308W, όπου N=North(+), S=South(-), E=East(-) και W=West(+).\\
Γράψτε πρόγραμμα assembly MIPS που συνεχώς:
\begin{enumerate}[i.]
    \itemsep0em
    \item Θα εκτυπώνει: "Give geographic coordinates as a string of $11$ chars (e.g. 3827N02308E): ",
    \item Θα αλλάζει γραμμή,
    \item Θα δέχεται ως είσοδο μία συμβολοσειρά $11$ χαρακτήρων \underline{ακριβώς} και θα την εκτυπώνει,
    \item Θα αλλάζει γραμμή,
    \item Θα εκτυπώνει: "You have given: ",
    \item Στη συνέχεια και σύμφωνα με την αρχική περιγραφή και σχετικά με τα -/+ θα μετατρέπει τους
    δύο αριθμούς που περιέχονται στο 11/ψήφιο σε πραγματικούς απλής ακρίβειας, όπως το
    παρακάτω παράδειγμα για τον 3827N02308E: +38.27 και -23.08.\\
    Λάβετε υπόψη ότι το πρώτο μέρος μπορεί να είναι από -90.00 έως +90.00, ενώ το δεύτερο μέρος
    από -180.00 έως 180.00.
    \item στη συνέχεια θα τους εκτυπώνει. Πρώτα θα εκτυπώνει σχετικά με το North ή South, π.χ.
    "Latitude 38.27",
    \item Θα αλλάζει γραμμή,
    \item και τέλος θα εκτυπώνει σε σχέση με West ή East, π.χ. "Longitude -23.08"
\end{enumerate}
Το πρόγραμμα θα τερματίζει όταν δοθεί ως είσοδος αντί ενός 11/ψήφιου το Enter.
\newpage
\section*{Exam 4}
Ο πιλότος ενός σύγχρονου αεροπλάνου αερογραμμών περνάει στο FMS (Flight Management System)
την ταχύτητα με την οποία θα πετάξει το αεροπλάνο κατά το ταξίδι ως έναν 5/ψήφιο της μορφής
N0085 ή K0085 ή M0085, όπου N=Knots (κόμβοι, άρα 85 Knots), K=Km/h (85 Km/h),
M=Mach number (Mach 0.85).\\
Γράψτε πρόγραμμα assembly MIPS που συνεχώς:
\begin{enumerate}[i.]
    \itemsep0em
    \item Θα εκτυπώνει: "Give cruise speed (e.g. N0085, K0085, M0085): ",
    \item Θα αλλάζει γραμμή,
    \item Θα δέχεται ως είσοδο μία συμβολοσειρά $5$ χαρακτήρων \underline{ακριβώς} και θα την εκτυπώνει,
    \item Θα αλλάζει γραμμή,
    \item Θα εκτυπώνει: "Selected cruise speed: ",
    \item Στη συνέχεια θα εκτυπώνει την επιλεγμένη ταχύτητα ταξιδιού σύμφωνα με την αρχική
    περιγραφή και θα μετατρέπει τον αριθμό που περιέχεται στο 5/ψήφιο σε πραγματικό απλής
    ακρίβειας, όπως τα παρακάτω παραδείγματα για τα N0085, K0085, M0085:\\
    85.00 αν πρόκειται για Knots ή Km/h και 0.85 αν πρόκειτε για Mach number.
    \item στη συνέχεια θα τους εκτυπώνει ανάλογα με την είσοδο. Δηλαδή:
        \begin{enumerate}[a.]
            \item για N0085, θα εκτυπώσει "85.00 Knots",
            \item για K0085, θα εκτυπώσει "85.00 Km/h", και τέλος
            \item για M0085, θα εκτυπώσει "Mach 0.85"
        \end{enumerate}
\end{enumerate}
Το πρόγραμμα θα τερματίζει όταν δοθεί ως είσοδος αντί ενός 5/ψήφιου το Enter.
\section*{Exam 8}
Γράψτε πρόγραμμα assembly MIPS που:
\begin{enumerate}[i.]
    \itemsep0em
    \item Θα εκτυπώνει: "Enter lastname and firstname as a string of $30$ chars: ",
    \item Θα αλλάζει γραμμή,
    \item Θα δέχεται ως είσοδο το ονοματεπώνυμο του φοιτητή ως μία συμβολοσειρά $30$ το πολύ χαρακτήρων και θα την εκτυπώνει,
    \item Θα αλλάζει γραμμή,
    \item Θα εκτυπώνει: "Enter an integer between 30 and 130: ",
    \item Θα αλλάζει γραμμή,
    \item Θα δέχεται ως είσοδο έναν ακέραιο $n$, μεταξύ 30 και 130 και θα τον εκτυπώνει.
    \item Θα αλλάζει γραμμή,
    \item Θα υπολογίζει το παρακάτω άθροισμα ως πραγματικό αριθμό απλής ακρίβειας. Ο
    αριθμός των διαδοχικών όρων/τιμών που πρέπει να υπολογιστούν θα πρέπει να είναι ίσος
    με τον ακέραιο που αναγνώστικε ήδη (μεταξύ 30 και 130).
    \begin{align}
        1+\frac{1}{4}+\frac{1\times3}{4\times8}
        +\frac{1\times3\times5}{4\times8\times12}+\hdots\nonumber
    \end{align}
    \item Θα εκτυπώνει έναν-έναν τους $n$ όρους του παραπάνω αθροίσματος ως πραγματικούς απλής ακρίβειας,
    \item Θα αλλάζει γραμμή,
    \item Θα εκτυπώνει: "Sum of $n$ fractions= ",
    \item Θα εκτυπώνει το άθροισμα των $n$ όρων.
    \item Θα τερματίζει.
\end{enumerate}
\end{document}